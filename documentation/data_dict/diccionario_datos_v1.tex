\documentclass[12pt,a4paper]{article}

\usepackage{amsmath}				% For use of  a large range of formulas, commands, and symbols
\usepackage[utf8]{inputenc}			% Specifies the encoding. (Zeichenkodierung, bindet Sonderzeichen mit ein)
\usepackage[T1]{fontenc}			% Include the accented characters as individual glyphs(ö is one single glyph, not an o with an added accent)
\usepackage[spanish]{babel}			% Specifies the language of doucment (translates "Table of Contents"...)
\usepackage{graphicx}				% For including graphics
\graphicspath{{images}}
\usepackage{subfigure}				% For including "sub"figures
\usepackage{lscape}					% Rotates text within environment by 90 degrees.
\usepackage{pst-plot, pstricks}		% plot­ting of data (typ­i­cally from ex­ter­nal files), plotting lines, rectangles...
\usepackage{fancybox,amssymb,color}	% fancybox: frames, rotations - amssymb: extension for amsmath - color: set the font color, text background, or page
\usepackage{setspace}				% allows to specifiy the space between lines
\usepackage{booktabs} 				% For prettier tables
\onehalfspacing
\pagestyle{plain}
\usepackage{enumerate}
\usepackage[hidelinks]{hyperref}
\setlength{\parindent}{1em}
\setlength{\parskip}{1ex}
\usepackage{geometry}   		 % definition of the page layout
\geometry{
	a4paper,					 % format
	left=30mm,					 % left margin
	right=30mm,					 % right margin
	top=25mm,					 % distance upwards
	bottom=25mm,				 % distance downwards
	includehead,				 % distance from upwards till the page header
}


\begin{document}

\begin{center}
	\textbf{Diccionario de Datos} \\
	Ultima vez editado: \today
\end{center}
\vspace{10mm}

\textbf{Nivel de desagregación de los datos}
\begin{enumerate}
	\item A nivel vivienda: DIRECTORIO
	\item A nivel hogar: DIRECTORIO + SECUENCIA\_P
	\item A nivel persona: DIRECTORIO + SECUENCIA\_P + ORDEN
\end{enumerate}

Una vivienda puede tener múltiples hogares. Se entiende el hogar como una persona o grupo de personas que ocupan la totalidad o parte de una vivienda y que se han asociado para compartir espacio, comida, descanso, etcétera... Los datos están desagregados al nivel de \textbf{PERSONA}.

Las bases de datos se componen de 23 variables:

\begin{enumerate}
	\item \textbf{DIRECTORIO}: Vivienda
	\item \textbf{SECUENCIA\_P}: Hogar 
	\item \textbf{ORDEN}: Persona en el hogar
	\item \textbf{conformeContrato}: Variable de satisfacción laboral binaria que toma dos valores:
	\begin{itemize}
		\item \textbf{0} Si esta conforme con el contrato que tiene
		\item \textbf{1} En otro caso
	\end{itemize}
	\item \textbf{quiereCambiar}: Variable de satisfacción laboral binaria que toma dos valores:
	\begin{itemize}
		\item \textbf{0} Si desea cambiar el trabajo que tiene actualmente
		\item \textbf{1} En otro caso
	\end{itemize}
	\item \textbf{conformeTrabajo}: Variable de satisfacción laboral binaria que toma dos valores:
	\begin{itemize}
		\item \textbf{0} Si no esta satisfecho con su trabajo actual
		\item \textbf{1} En otro caso
	\end{itemize}
	\item \textbf{ingresosTrabajo}: Variable continua que indica los ingresos mensuales del encuestado por concepto de actividades laborales
	\item \textbf{actividadEconomica}: Variable categórica basada en la clasificación del \textit{Código Industrial Internacional Uniforme rev. 3} de dos dígitos. Toma los siguientes valores:
	\begin{itemize}
		\item \textbf{AtB}: Agricultura, cacería, silvicultura y pesca
		\item \textbf{C}: Minería y extracción
		\item \textbf{D}: Manufacturas
		\item \textbf{E}: Electricidad, Gas y Agua
		\item \textbf{F}: Construcción
		\item \textbf{GtH}: Comercio, hoteles y restaurantes
		\item \textbf{I}: Transporte, almacenamiento y comunicaciones
		\item \textbf{JtK}: Finanzas, seguros, bienes raíces, y servicios de negocio
		\item \textbf{LtQ}: Comunidad social y servicios personales
		\item \textbf{Otro}: Otra actividad o actividades no clasificadas
	\end{itemize}
		\item \textbf{horasTrabajo}: Variable continua que indica las horas semanales dedicadas por el encuestado al trabajo normalmente
		\item \textbf{subempleadoHoras}: Variable de subempleo binaria que toma dos valores:
		\begin{itemize}
			\item \textbf{0} No reporta subempleo por horas, es decir, no desea trabajar mas horas
			\item \textbf{1} En otro caso
		\end{itemize}
		\item \textbf{subempleadoIngresos}: Variable de subempleo binaria que toma dos valores:
		\begin{itemize}
			\item \textbf{0} No reporta subempleo por ingresos, es decir, no desea cambiar su empleo para mejorar sus ingresos
			\item \textbf{1} En otro caso
		\end{itemize}
		\item \textbf{cotizaPension}: Variable de seguridad social binaria que toma dos valores:
		\begin{itemize}
			\item \textbf{0} Se encuentra cotizando en un fondo de pensiones
			\item \textbf{1} En otro caso
		\end{itemize}
		\item \textbf{posicionOcupacional}: Variable categórica que toma distintos valores
		\begin{itemize}
			\item \textbf{1}: Empleado convencional
			\item \textbf{2}: Obrero o empleado del gobierno
			\item \textbf{3}: Trabajador por cuenta propia
			\item \textbf{4}: Patrón o empleador
			\item \textbf{5}: Otro tipo
		\end{itemize}
		\item \textbf{region}: Variable espacial categórica no numérica que toma distintos valores basado en la clasificación DIVIPOLA del DANE
		\begin{itemize}
			\item \textbf{Caribe}: Atlántico, Bolívar, Córdoba, Magdalena, César, La Guajira, Sucre
			\item \textbf{Central}: Antioquia, Boyacá, Caldas, Cundinamarca, Huila, Norte de Santander, Quindío, Risaralda, Santander, Tolima, Bogotá DC.
			\item \textbf{Pacifica}: Cauca, Chocó, Nariño,	Valle Del Cauca.
			\item \textbf{Oriental}: Meta, Caquetá
		\end{itemize}
		\item \textbf{segmentoEdad}: Variable  categórica que toma distintos valores basado en la edad:
		\begin{itemize}
			\item \textbf{1}: Igual o menor a 30 años
			\item \textbf{2}: Entre 30 y 40 años
			\item \textbf{3}: Entre 40 y 50 años
			\item \textbf{4}: Entre 50 y 60 años
			\item \textbf{5}: Mas de 60 años
		\end{itemize}
		\item \textbf{genero}: Variable binaria que toma dos valores:
		\begin{itemize}
			\item \textbf{1} Es mujer
			\item \textbf{0} En otro caso
		\end{itemize}
		\item \textbf{tienePareja}: Variable binaria que toma dos valores:
		\begin{itemize}
			\item \textbf{0} No tiene pareja
			\item \textbf{1} En otro caso
		\end{itemize}
		\item \textbf{segmentoEducativo}: Variable  categórica que toma distintos valores basado en el nivel educativo:
		\begin{itemize}
			\item \textbf{1}: Básica primaria o ninguna
			\item \textbf{2}: Secundaria
			\item \textbf{3}: Educación superior
		\end{itemize}
		\item \textbf{afiliadoSalud}: Variable binaria que toma dos valores:
		\begin{itemize}
			\item \textbf{0} Esta afiliado en alguna entidad de seguridad social en salud
			\item \textbf{1} En otro caso
		\end{itemize}
		\item \textbf{horasHogar}: Variable numerica continua que indica las horas semanales dedicadas a los oficios del hogar
		\item \textbf{ingresosTransferencias}: Variable numerica continua que indica el monto mensual recibido por concepto de transferencias monetarias
		\item \textbf{ingresosOtros}: Variable numerica continua que indica el monto mensual recibido por concepto de otros ingresos. Pueden ser dividendos, intereses, bienes raíces, venta de activos, pensión de vejez o alimenticia, etcetera...
		\item \textbf{maxTransf} (\textit{OPCIONAL}): Variable que verifica cual es el mayor monto recibido por concepto de transferencias
		\item \textbf{DSI}: Variable binaria recodificada que toma dos valores:
		\begin{itemize}
			\item \textbf{1} Si en el hogar (secuencia\_p) de la persona encuestada, que responde positivamente en las preguntas de el modulo ocupados sobre si esta empleado, hay al menos una persona que depende económicamente
			\item \textbf{0} En otro caso
		\end{itemize}
\end{enumerate}

Todos los datos provienen de la Gran Encuesta Integrada de Hogares para \textbf{\href{https://microdatos.dane.gov.co/index.php/catalog/77/study-description}{2012}} y  \textbf{\href{https://microdatos.dane.gov.co/index.php/catalog/701/study-description}{2021}}.

\vfill

\noindent Contacto: \href{mailto:jtabordaj@uninorte.edu.co}{jtabordaj@uninorte.edu.co} \\
Repositorio: \textbf{\href{https://github.com/jtabordaj/research_perilla}{GitHub}}

\end{document}