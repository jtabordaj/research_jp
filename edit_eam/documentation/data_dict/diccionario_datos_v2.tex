\documentclass[12pt,a4paper]{article}

\usepackage{amsmath}				% For use of  a large range of formulas, commands, and symbols
\usepackage[utf8]{inputenc}			% Specifies the encoding. (Zeichenkodierung, bindet Sonderzeichen mit ein)
\usepackage[T1]{fontenc}			% Include the accented characters as individual glyphs(ö is one single glyph, not an o with an added accent)
\usepackage[spanish]{babel}			% Specifies the language of doucment (translates "Table of Contents"...)
\usepackage{graphicx}				% For including graphics
\graphicspath{{images}}
\usepackage{subfigure}				% For including "sub"figures
\usepackage{lscape}					% Rotates text within environment by 90 degrees.
\usepackage{pst-plot, pstricks}		% plot­ting of data (typ­i­cally from ex­ter­nal files), plotting lines, rectangles...
\usepackage{fancybox,amssymb,color}	% fancybox: frames, rotations - amssymb: extension for amsmath - color: set the font color, text background, or page
\usepackage{setspace}				% allows to specifiy the space between lines
\usepackage{booktabs} 				% For prettier tables
\onehalfspacing
\pagestyle{plain}
\usepackage{enumerate}
\usepackage[hidelinks]{hyperref}
\setlength{\parindent}{1em}
\setlength{\parskip}{1ex}
\usepackage{geometry}   		 % definition of the page layout
\geometry{
	a4paper,					 % format
	left=30mm,					 % left margin
	right=30mm,					 % right margin
	top=25mm,					 % distance upwards
	bottom=25mm,				 % distance downwards
	includehead,				 % distance from upwards till the page header
}


\begin{document}

\begin{center}
	\textbf{Diccionario de Datos} \\
	Ultima vez editado: \today
\end{center}
\vspace{10mm}

\textbf{Nivel de desagregación de los datos}
\begin{enumerate}
	\item A nivel firma: NORDEMP
\end{enumerate}

Si un dato tiene mas de una observacion en la misma encuesta, pues se captura sobre dos espacios de tiempo distintos, la base de datos agrega un numero de dos o cuatro digitos que corresponde al periodo de tiempo. 

\begin{center}
	\textbf{\textbf{Ejemplo}: valorAgregado18 indica que es el dato de valor agregado para 2018; totOcupados2017 indica que es el numero de ocupados totales para 2017.}
\end{center}

Los conceptos sobre tipo, taxonomía, tipologia y demás elementos relacionados a la innovación son tomados de la metodología oficial de la encuesta EDIT para 2017-2018.

Las bases de datos se componen de 103 variables, repartidas en varios ejes temáticos de interés:

\section{Tipo de innovación}

\begin{enumerate}
	\item innProceso: Variable numerica que indica el numero de innovaciones en métodos de producción, distribución, entrega, o sistemas logísticos introducidas por la firma en el periodo de la encuesta.
	\item innProducto: Variable numerica que indica el numero de innovaciones en canales para promoción y venta, o modificaciones significativas en el empaque o diseño del producto introducidas por la firma en el periodo de la encuesta.
	\item innOrganizacional: Variable numerica que indica el numero de innovaciones en sistema de gestión del conocimiento, organización del lugar de trabajo, o en la gestión de las relaciones externas de la empresa introducidas por la firma en el periodo de la encuesta.
\end{enumerate}

\section{Impacto de la innovación}

\begin{enumerate}
	\item mejoraBS: Variable binaria que toma el valor de 1 si la innovación tuvo impacto en la mejora de calidad de bienes o servicios, y 0 en otro caso.
	\item aumentaBS: Variable binaria que toma el valor de 1 si la innovación tuvo impacto en ampliar la gama de bienes o servicios ofrecidos, y 0 en otro caso.
	\item mantienePosicion: Variable binaria que toma el valor de 1 si la innovación tuvo impacto en mantener la posición de mercado o ingresar a uno nuevo, y 0 en otro caso.
	\item aumentaProductividad: Variable binaria que toma el valor de 1 si la innovación tuvo impacto en aumentar la productividad, y 0 en otro caso.
	\item reduceCostoFactores: Variable binaria que toma el valor de 1 si la innovación tuvo impacto en reducir los costos asociados a factores productivos, y 0 en otro caso.
	\item reduceCostoLogistico: Variable binaria que toma el valor de 1 si la innovación tuvo impacto en reducir los costos asociados a logística de la empresa, y 0 en otro caso.
	\item reduceCostoOtros: Variable binaria que toma el valor de 1 si la innovación tuvo impacto en reducir otros costos, como tributarios, de reciclaje o regulatorios, y 0 en otro caso.
\end{enumerate}

\section{Taxonomía de la innovación}

\begin{enumerate}
	\item innRadical: Variable numerica que indica el numero de bienes o servicios nuevos introducidos por la empresa al mercado.
	\item innIncremental: Variable numerica que indica el numero de bienes o servicios significativamente mejorados introducidos por la empresa en el mercado.
\end{enumerate}

\section{Tipo de empresa innovadora}

\begin{enumerate}
	\item innovadorEstricto: Variable binaria que toma el valor de 1 si la firma, en el período de referencia de la encuesta, obtuvo al menos un bien o servicio nuevo o significativamente mejorado para el mercado internacional. 0 en otros casos.
	\item innovadorAmplio: Variable binaria que toma el valor de 1 si la firma, en el período de referencia, obtuvo al menos un bien o servicio nuevo o significativamente mejorado para el mercado nacional o un bien o servicio nuevo o mejorado para la empresa, o que implementaron nuevos o significativamente mejorados métodos de prestación de servicios, producción, distribución, entrega, o sistemas logísticos o una forma organizacional o de comercialización nueva. 0 en otros casos
	\item innovadorPotencial: Variable binaria que toma el valor de 1 si la firma, en el período de referencia, no había obtenido ninguna innovación, pero que reportaron tener en proceso o haber abandonado algún proyecto de innovación. 0 en otros casos.
	\item noInnovadora: Variable binaria que toma el valor de 1 si la firma, en el período de referencia, no obtuvieron innovaciones, ni reportaron tener en proceso, o haber abandonado, algún proyecto para la obtención de Innovaciones. 0 en otros casos.
	\item otroTipo: Variable binaria que toma el valor de 1 si la firma, en el período de referencia, no reporta estar en ninguna de las tipologias anteriores. 0 en otros casos.
\end{enumerate}

\section{Uso de TICs}

\begin{enumerate}
	\item usaTICs: Variable binaria que toma el valor de 1 si la firma, en el período de referencia, hizo uso de Tecnologías de Información y Telecomunicaciones, 0 en otro caso.
\end{enumerate}

\section{Networking de la innovacion}

\begin{enumerate}
	\item networkNoEmpresa: Variable binaria que toma el valor de 1 si la firma, en el período de referencia, coopero con clientes, consultores, universidades, centros de desarrollo tecnológico, centros de investigación autónomos, parques tecnológicos, centros regionales de productividad, organizaciones no gubernamentales o el gobierno para el desarrollo de actividades innovadoras. 0 en otro caso.
	\item networkEmpresas: Variable binaria que toma el valor de 1 si la firma, en el período de referencia, coopero con empresas del mismo grupo o conglomerado, proveedores o competidores para el desarrollo de actividades innovadoras. 0 en otro caso.
	\item modificaOtros: Variable binaria que toma el valor de 1 si la firma, en el período de referencia, hizo modificaciones a innovaciones de otras empresas. 0 en otro caso.
	\item noNetwork:  Variable binaria que toma el valor de 1 si la firma no coopero con ninguno de los agentes anterior mencionados. 0 en otro caso.
\end{enumerate}

\section{Propiedad de la empresa}

\begin{enumerate}
	\item propietarioActual: Variable categorica que relaciona el parentesco de el mayor accionista o actual propietario de la empresa con el fundador, puede tomar 3 valores:
		 \begin{enumerate}
		 	\item 1 = El mayor accionista o propietario actual es el fundador.
		 	\item 2 = El mayor accionista o propietario actual es familiar del fundador.
		 	\item 3 = El mayor accionista o propietario actual es otra persona o institución.
		 \end{enumerate}
\end{enumerate}

\section{Personal ocupado}

\begin{enumerate}
	\item totOcupados: Variable numerica que indica el total de personal ocupado en determinado periodo de tiempo. \textbf{Se reporta por año}.
	\item catOcupados: Variable categórica que organiza las empresas por el numero de ocupados que tiene en determinado periodo de tiempo, puede tomar 4 valores (\textbf{se reporta por año}):
	\begin{enumerate}
		\item 1 = Menos de 10.
		\item 2 = Entre 10 y 50.
		\item 3 = Entre 50 y 200.
		\item 4 = Mas de 200.
	\end{enumerate}
	\item ocupadosACTI: Variable numerica que indica el total de personal ocupado en actividades innovadoras sobre determinado periodo de tiempo. \textbf{Se reporta por año}.
	\item ratioActi: Variable numerica que indica la razon de personal ocupado en actividades innovadoras sobre los ocupados totales. La formula de calculo es $ oACTI/ oTotal $, donde oActi son ocupados en ACTI, y oTotal son ocupados totales. \textbf{Se reporta por año}.
\end{enumerate}

\section{Actividad Económica}

\begin{enumerate}
	\item ciiu4Digitos: Variable categórica numerica que indica el código de Clasificación Industrial Internacional Uniforme (ISIC) de la empresa. En este caso, para la sección C (manufacturas). El código tiene especificidad de 4 dígitos, donde el primero indica la sección, el segundo el sector y los dos últimos el subsector. Para mas informacion:  \href{https://www.dane.gov.co/files/sen/nomenclatura/ciiu/CIIU_Rev_4_AC2020.pdf}{\textbf{Aqui}}.
\end{enumerate}

\section{Localización espacial}

\begin{enumerate}
	\item dptoDivipola: Variable categórica numerica que indica la ubicación de la firma según la clasificación \href{https://geoportal.dane.gov.co/geovisores/territorio/consulta-divipola-division-politico-administrativa-de-colombia/}{\textbf{DIVIPOLA}} del DANE para departamentos y municipios de Colombia. 
	\item region: Variable categórica que indica la region de Colombia sobre la cual esta ubicada la firma, puede adoptar los siguientes valores:
		\begin{enumerate}
			\item Caribe: Región Caribe
			\item Central: Región Central, incluyendo Bogotá D.C 
			\item Pacifico: Región pacifica
			\item Orinoquia: Región de la Orinoquia
		\end{enumerate}
\end{enumerate} 

\section{Métricas industriales}

\begin{enumerate}
	\item valorAgregado: Variable numerica que indica el valor agregado según la metodología del DANE, el cual la entiende como la diferencia entre la producción bruta y el valor intermedio. Esta expresada en miles de millones de pesos. \textbf{Se reporta por año}.
	\item productividadL: Variable numerica que indica la productividad del trabajo, definida como la razón de valor agregado sobre el total de empleados en la firma. La formula de calculo viene definida por $ VA / totO $ donde VA es el valor agregado y totO es el total de ocupados. Esta expresada en miles de millones de pesos. \textbf{Se reporta por año}.
	\item ratioX: Variable numerica que indica la razón de venta hecha al extranjero, es decir, que proporción de las ventas totales fueron ventas hechas al extranjero. La formula de calculo es definida como $ VX / VEN $, donde VX es ventas al extranjero y VEN es ventas totales. \textbf{Se reporta por año}.
	\item ratioM: Variable numerica que indica la razón de compra de materias primas hechas al extranjero, es decir, relativo a las ventas totales, cuanta materia prima se compro al extranjero. La formula de calculo es definida como $ CX/VEN $, donde CX es compra al extranjero y VEN son ventas totales. \textbf{Se reporta por año}
	\item inversión: Variable numerica que indica la inversion de la firma, definida como $ ACTI/VA*100 $, donde ACTI son actividades innovadoras y VA es valor agregado. \textbf{Se reporta por año}.
\end{enumerate}

\section{Características del capital humano}

\begin{enumerate}
	\item educadosColegio: Variable numerica que indica el numero de ocupados que tienen educacion primaria o bachillerato. \textbf{Se reporta por año}.
	\item educadosSuperior: Variable numerica que indica el numero de ocupados que tienen educación superior (pre grado, técnico o tecnologo).  \textbf{Se reporta por año}.
	\item educadosPosgrado: Variable numerica que indica el numero de ocupados que tienen educación de posgrado (maestría, especialización, doctorado, pos-doctorado).  \textbf{Se reporta por año}.
	\item educadosOtros: Variable numerica que indica el numero de ocupados que no tienen ninguno de los tres niveles de educación formal descritos anteriormente. \textbf{Se reporta por año}.
	\item ocupadosProduccion: Variable numerica que indica el numero de ocupados, cuya responsabilidad en la empresa es actividades de producción. 
	\item ratioProduccion: Variable numerica que indica la proporción de ocupados dedicados a actividades de producción sobre el total de ocupados, la formula de calculo es $ ocupA / ocupT $, donde ocupA es la cantidad de ocupados en la actividad y ocupT son los ocupados totales.
	\item ocupadosID: Variable numerica que indica el numero de ocupados, cuya responsabilidad en la empresa es actividades de I+D. 
	\item ratioID: Variable numerica que indica la proporción de ocupados dedicados a actividades de I+D sobre el total de ocupados, la formula de calculo es $ ocupA / ocupT $, donde ocupA es la cantidad de ocupados en la actividad y ocupT son los ocupados totales.
	\item ocupadosAdmin: Variable numerica que indica el numero de ocupados, cuya responsabilidad en la empresa es actividades de administración. 
	\item ratioAdmin: Variable numerica que indica la proporción de ocupados dedicados a actividades de administración sobre el total de ocupados, la formula de calculo es $ ocupA / ocupT $, donde ocupA es la cantidad de ocupados en la actividad y ocupT son los ocupados totales.
	\item ocupadosMarketing: Variable numerica que indica el numero de ocupados, cuya responsabilidad en la empresa es actividades de mercadeo. 
	\item ratioMarketing: Variable numerica que indica la proporción de ocupados dedicados a actividades de mercadeo sobre el total de ocupados, la formula de calculo es $ ocupA / ocupT $, donde ocupA es la cantidad de ocupados en la actividad y ocupT son los ocupados totales.
	\item hayMujeres: Variable binaria que toma el valor de 1 si hay mujeres entre los ocupados dedicados a actividades innovadoras, 0 en otro caso.
\end{enumerate}

\section{Fuentes de financiamiento}

\begin{enumerate}
	\item recursosPropios: Variable numerica que indica la cantidad de recursos propios invertidos por la firma en actividades innovadoras. En miles de pesos. \textbf{Se reporta por año}.
	\item recursosBanca: Variable numerica que indica la cantidad de recursos de la banca invertidos por la firma en actividades innovadoras. En miles de pesos. \textbf{Se reporta por año}.
	\item recursosConglomerado: Variable numerica que indica la cantidad de recursos del grupo empresarial o conglomerado invertidos por la firma en actividades innovadoras. En miles de pesos. \textbf{Se reporta por año}.
	\item recursosPublicos: Variable numerica que indica la cantidad de recursos públicos invertidos por la firma en actividades innovadoras. En miles de pesos. \textbf{Se reporta por año}.
	\item recursosEmpresas: Variable numerica que indica la cantidad de recursos de otras empresas invertidos por la firma en actividades innovadoras. En miles de pesos. \textbf{Se reporta por año}.
	\item recursosOtros: Variable numerica que indica la cantidad de recursos de otras fuentes (fondos de capital privado, donaciones) invertidos por la firma en actividades innovadoras. En miles de pesos. \textbf{Se reporta por año}.
	\item contratoEstado: Variable binaria que toma el valor de 1 si la firma celebro contratos con el estado en el periodo de interés de la encuesta, 0 en otros casos.
\end{enumerate}

\section{Sistema Nacional de Innovación}

\begin{enumerate}
	\item sniSENA:  Variable binaria que toma el valor de 1 si la firma, durante el periodo de interés de la encuesta, coopero con el SENA para el desarrollo de actividades innovadoras, 0 en otro caso.
	\item sniCamaraComercio:  Variable binaria que toma el valor de 1 si la firma, durante el periodo de interés de la encuesta, coopero con la Cámara de Comercio para el desarrollo de actividades innovadoras, 0 en otro caso.
	\item sniICONTEC:  Variable binaria que toma el valor de 1 si la firma, durante el periodo de interés de la encuesta, coopero con ICONTEC para el desarrollo de actividades innovadoras, 0 en otro caso.
	\item sniUnivSNCTI:  Variable binaria que toma el valor de 1 si la firma, durante el periodo de interés de la encuesta, coopero con una Universidad registrada en el Sistema Nacional de Ciencia, Tecnología e Innovación para el desarrollo de actividades innovadoras, 0 en otro caso.
	\item sniSIC:  Variable binaria que toma el valor de 1 si la firma, durante el periodo de interés de la encuesta, coopero con la Superintendencia de Industria y Comercio para el desarrollo de actividades innovadoras, 0 en otro caso.
	\item sniMinciencias:  Variable binaria que toma el valor de 1 si la firma, durante el periodo de interés de la encuesta, coopero con el Ministerio de Ciencias para el desarrollo de actividades innovadoras, 0 en otro caso.
	\item sniConsultores:  Variable binaria que toma el valor de 1 si la firma, durante el periodo de interés de la encuesta, coopero con consultores para el desarrollo de actividades innovadoras, 0 en otro caso.
	\item sniProcolombia:  Variable binaria que toma el valor de 1 si la firma, durante el periodo de interés de la encuesta, coopero con Procolombia para el desarrollo de actividades innovadoras, 0 en otro caso.
	\item sniINNPulsa:  Variable binaria que toma el valor de 1 si la firma, durante el periodo de interés de la encuesta, coopero con INNPulsa para el desarrollo de actividades innovadoras, 0 en otro caso.
\end{enumerate}

\section{Obstáculos de recursos}

\begin{enumerate}
	\item obsRecursosPropios: Variable binaria que toma el valor de 1 si, durante el periodo de interés de la encuesta, la falta de recursos propios fue un obstáculo para la firma en el desarrollo de actividades innovadoras, 0 en otro caso.
	\item obsFinancExterno: Variable binaria que toma el valor de 1 si, durante el periodo de interés de la encuesta, la dificultad para obtener financiamiento externo fue un obstáculo para la firma en el desarrollo de actividades innovadoras, 0 en otro caso.
\end{enumerate}

\section{Obstáculos legales o técnicos}

\begin{enumerate}
	\item obsApoyoPublico: Variable binaria que toma el valor de 1 si, durante el periodo de interés de la encuesta, la falta de financiamiento publico o información sobre mecanismos públicos fue un obstáculo para la firma en el desarrollo de actividades innovadoras, 0 en otro caso.
	\item obsInformacion: Variable binaria que toma el valor de 1 si, durante el periodo de interés de la encuesta, la falta de información sobre el mercado o tecnologías relevantes fue un obstáculo para la firma en el desarrollo de actividades innovadoras, 0 en otro caso.
	\item obsPersonal: Variable binaria que toma el valor de 1 si, durante el periodo de interés de la encuesta, la falta de personal calificado fue un obstáculo para la firma en el desarrollo de actividades innovadoras, 0 en otro caso.
	\item obsRegulaciones: Variable binaria que toma el valor de 1 si, durante el periodo de interés de la encuesta, el cumplimiento de regulaciones y normativas similares fue un obstáculo para la firma en el desarrollo de actividades innovadoras, 0 en otro caso.
	\item obsCooperacion: Variable binaria que toma el valor de 1 si, durante el periodo de interés de la encuesta, la falta de oportunidades para cooperar con otras empresas fue un obstáculo para la firma en el desarrollo de actividades innovadoras, 0 en otro caso.
	\item obsDerechosProp: Variable binaria que toma el valor de 1 si, durante el periodo de interés de la encuesta, la insuficiencia del sistema de derechos de propiedad fue un obstáculo para la firma en el desarrollo de actividades innovadoras, 0 en otro caso.
	\item obsInspeccion: Variable binaria que toma el valor de 1 si, durante el periodo de interés de la encuesta, la escasez de servicios de inspección fue un obstáculo para la firma en el desarrollo de actividades innovadoras, 0 en otro caso.
	\item obsTramite: Variable binaria que toma el valor de 1 si, durante el periodo de interés de la encuesta, el tramite excesivo fue un obstáculo para la firma en el desarrollo de actividades innovadoras, 0 en otro caso.
	\item obsIntermediacion: Variable binaria que toma el valor de 1 si, durante el periodo de interés de la encuesta, la demora de intermediación entre la banca comercial y lineas de crédito publico fue obstáculo para la firma en el desarrollo de actividades innovadoras, 0 en otro caso.
	\item obsRequisitos: Variable binaria que toma el valor de 1 si, durante el periodo de interés de la encuesta, el cumplimiento de requisitos fue obstáculo para la firma en el desarrollo de actividades innovadoras, 0 en otro caso.
	\item obsFinAtractiva: Variable binaria que toma el valor de 1 si, durante el periodo de interés de la encuesta, la falta de oportunidades de financiamiento atractivas fue obstáculo para la firma en el desarrollo de actividades innovadoras, 0 en otro caso.
\end{enumerate}

\section{Obstáculos de incertidumbre}

\begin{enumerate}
	\item obsDemanda: Variable binaria que toma el valor de 1 si, durante el periodo de interés de la encuesta, la incertidumbre sobre la demanda del bien o servicio innovador fue obstáculo  para la firma en el desarrollo de actividades innovadoras, 0 en otro caso.
	\item obsTecnica: Variable binaria que toma el valor de 1 si, durante el periodo de interés de la encuesta, la incertidumbre sobre el éxito de la ejecución técnica del proyecto fue obstáculo  para la firma en el desarrollo de actividades innovadoras, 0 en otro caso.
	\item  obsImitacion: Variable binaria que toma el valor de 1 si, durante el periodo de interés de la encuesta, la incertidumbre sobre potenciales imitaciones al bien o servicio innovador fue obstáculo  para la firma en el desarrollo de actividades innovadoras, 0 en otro caso.
	\item  obsRentabilidad: Variable binaria que toma el valor de 1 si, durante el periodo de interés de la encuesta, la incertidumbre sobre rentabilidad del proyecto fue obstáculo  para la firma en el desarrollo de actividades innovadoras, 0 en otro caso.
\end{enumerate}

\section{Incentivos a la productividad}

\begin{enumerate}
	\item bonosGerenciales: Variable binaria que toma el valor de 1 si, durante el periodo de interés de la encuesta, la empresa otorgo bonos a cargos gerenciales como incentivo al cumplimiento de metas u otros objetivos, 0 en otro caso.
	\item bonosNoGerenciales: Variable binaria que toma el valor de 1 si, durante el periodo de interés de la encuesta, la empresa otorgo bonos a cargos no gerenciales como incentivo al cumplimiento de metas u otros objetivos, 0 en otro caso.
	\item ascensosGerenciales: Variable binaria que toma el valor de 1 si, durante el periodo de interés de la encuesta, la empresa asciende cargos gerenciales, 0 en otro caso.
	\item ascensosNoGerenciales: Variable binaria que toma el valor de 1 si, durante el periodo de interés de la encuesta, la empresa asciende cargos no gerenciales, 0 en otro caso.
	\item metasProduccion: Variable binaria que toma el valor de 1 si, durante el periodo de interés de la encuesta, la empresa uso metas de producción, 0 en otro caso.
	\item indicadoresDesemp: Variable binaria que toma el valor de 1 si, durante el periodo de interés de la encuesta, la empresa uso indicadores de desempeño, 0 en otro caso.
\end{enumerate}


Todos los datos provienen de la Encuesta de Desarrollo e Innovación Tecnológica, y la Encuesta Anual Manufacturera para los periodos 2017, 2018, 2019 y 2020. Para mas información, ver: \textbf{\href{https://www.dane.gov.co/index.php/estadisticas-por-tema/industria/encuesta-anual-manufacturera-enam}{EAM}} y  \textbf{\href{https://www.dane.gov.co/index.php/estadisticas-por-tema/tecnologia-e-innovacion/encuesta-de-desarrollo-e-innovacion-tecnologica-edit}{EDIT}}.

\vfill

\noindent Contacto: \href{mailto:jtabordaj@uninorte.edu.co}{jtabordaj@uninorte.edu.co} \\
Repositorio: \textbf{\href{https://github.com/jtabordaj/research_perilla}{GitHub}}


\end{document}